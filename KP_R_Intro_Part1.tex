\documentclass[]{article}
\usepackage{lmodern}
\usepackage{amssymb,amsmath}
\usepackage{ifxetex,ifluatex}
\usepackage{fixltx2e} % provides \textsubscript
\ifnum 0\ifxetex 1\fi\ifluatex 1\fi=0 % if pdftex
  \usepackage[T1]{fontenc}
  \usepackage[utf8]{inputenc}
\else % if luatex or xelatex
  \ifxetex
    \usepackage{mathspec}
  \else
    \usepackage{fontspec}
  \fi
  \defaultfontfeatures{Ligatures=TeX,Scale=MatchLowercase}
\fi
% use upquote if available, for straight quotes in verbatim environments
\IfFileExists{upquote.sty}{\usepackage{upquote}}{}
% use microtype if available
\IfFileExists{microtype.sty}{%
\usepackage{microtype}
\UseMicrotypeSet[protrusion]{basicmath} % disable protrusion for tt fonts
}{}
\usepackage[margin=1in]{geometry}
\usepackage{hyperref}
\hypersetup{unicode=true,
            pdftitle={KP R Intro, Part 1},
            pdfauthor={Nerissa Nance},
            pdfborder={0 0 0},
            breaklinks=true}
\urlstyle{same}  % don't use monospace font for urls
\usepackage{color}
\usepackage{fancyvrb}
\newcommand{\VerbBar}{|}
\newcommand{\VERB}{\Verb[commandchars=\\\{\}]}
\DefineVerbatimEnvironment{Highlighting}{Verbatim}{commandchars=\\\{\}}
% Add ',fontsize=\small' for more characters per line
\usepackage{framed}
\definecolor{shadecolor}{RGB}{248,248,248}
\newenvironment{Shaded}{\begin{snugshade}}{\end{snugshade}}
\newcommand{\KeywordTok}[1]{\textcolor[rgb]{0.13,0.29,0.53}{\textbf{#1}}}
\newcommand{\DataTypeTok}[1]{\textcolor[rgb]{0.13,0.29,0.53}{#1}}
\newcommand{\DecValTok}[1]{\textcolor[rgb]{0.00,0.00,0.81}{#1}}
\newcommand{\BaseNTok}[1]{\textcolor[rgb]{0.00,0.00,0.81}{#1}}
\newcommand{\FloatTok}[1]{\textcolor[rgb]{0.00,0.00,0.81}{#1}}
\newcommand{\ConstantTok}[1]{\textcolor[rgb]{0.00,0.00,0.00}{#1}}
\newcommand{\CharTok}[1]{\textcolor[rgb]{0.31,0.60,0.02}{#1}}
\newcommand{\SpecialCharTok}[1]{\textcolor[rgb]{0.00,0.00,0.00}{#1}}
\newcommand{\StringTok}[1]{\textcolor[rgb]{0.31,0.60,0.02}{#1}}
\newcommand{\VerbatimStringTok}[1]{\textcolor[rgb]{0.31,0.60,0.02}{#1}}
\newcommand{\SpecialStringTok}[1]{\textcolor[rgb]{0.31,0.60,0.02}{#1}}
\newcommand{\ImportTok}[1]{#1}
\newcommand{\CommentTok}[1]{\textcolor[rgb]{0.56,0.35,0.01}{\textit{#1}}}
\newcommand{\DocumentationTok}[1]{\textcolor[rgb]{0.56,0.35,0.01}{\textbf{\textit{#1}}}}
\newcommand{\AnnotationTok}[1]{\textcolor[rgb]{0.56,0.35,0.01}{\textbf{\textit{#1}}}}
\newcommand{\CommentVarTok}[1]{\textcolor[rgb]{0.56,0.35,0.01}{\textbf{\textit{#1}}}}
\newcommand{\OtherTok}[1]{\textcolor[rgb]{0.56,0.35,0.01}{#1}}
\newcommand{\FunctionTok}[1]{\textcolor[rgb]{0.00,0.00,0.00}{#1}}
\newcommand{\VariableTok}[1]{\textcolor[rgb]{0.00,0.00,0.00}{#1}}
\newcommand{\ControlFlowTok}[1]{\textcolor[rgb]{0.13,0.29,0.53}{\textbf{#1}}}
\newcommand{\OperatorTok}[1]{\textcolor[rgb]{0.81,0.36,0.00}{\textbf{#1}}}
\newcommand{\BuiltInTok}[1]{#1}
\newcommand{\ExtensionTok}[1]{#1}
\newcommand{\PreprocessorTok}[1]{\textcolor[rgb]{0.56,0.35,0.01}{\textit{#1}}}
\newcommand{\AttributeTok}[1]{\textcolor[rgb]{0.77,0.63,0.00}{#1}}
\newcommand{\RegionMarkerTok}[1]{#1}
\newcommand{\InformationTok}[1]{\textcolor[rgb]{0.56,0.35,0.01}{\textbf{\textit{#1}}}}
\newcommand{\WarningTok}[1]{\textcolor[rgb]{0.56,0.35,0.01}{\textbf{\textit{#1}}}}
\newcommand{\AlertTok}[1]{\textcolor[rgb]{0.94,0.16,0.16}{#1}}
\newcommand{\ErrorTok}[1]{\textcolor[rgb]{0.64,0.00,0.00}{\textbf{#1}}}
\newcommand{\NormalTok}[1]{#1}
\usepackage{longtable,booktabs}
\usepackage{graphicx,grffile}
\makeatletter
\def\maxwidth{\ifdim\Gin@nat@width>\linewidth\linewidth\else\Gin@nat@width\fi}
\def\maxheight{\ifdim\Gin@nat@height>\textheight\textheight\else\Gin@nat@height\fi}
\makeatother
% Scale images if necessary, so that they will not overflow the page
% margins by default, and it is still possible to overwrite the defaults
% using explicit options in \includegraphics[width, height, ...]{}
\setkeys{Gin}{width=\maxwidth,height=\maxheight,keepaspectratio}
\IfFileExists{parskip.sty}{%
\usepackage{parskip}
}{% else
\setlength{\parindent}{0pt}
\setlength{\parskip}{6pt plus 2pt minus 1pt}
}
\setlength{\emergencystretch}{3em}  % prevent overfull lines
\providecommand{\tightlist}{%
  \setlength{\itemsep}{0pt}\setlength{\parskip}{0pt}}
\setcounter{secnumdepth}{0}
% Redefines (sub)paragraphs to behave more like sections
\ifx\paragraph\undefined\else
\let\oldparagraph\paragraph
\renewcommand{\paragraph}[1]{\oldparagraph{#1}\mbox{}}
\fi
\ifx\subparagraph\undefined\else
\let\oldsubparagraph\subparagraph
\renewcommand{\subparagraph}[1]{\oldsubparagraph{#1}\mbox{}}
\fi

%%% Use protect on footnotes to avoid problems with footnotes in titles
\let\rmarkdownfootnote\footnote%
\def\footnote{\protect\rmarkdownfootnote}

%%% Change title format to be more compact
\usepackage{titling}

% Create subtitle command for use in maketitle
\newcommand{\subtitle}[1]{
  \posttitle{
    \begin{center}\large#1\end{center}
    }
}

\setlength{\droptitle}{-2em}
  \title{KP R Intro, Part 1}
  \pretitle{\vspace{\droptitle}\centering\huge}
  \posttitle{\par}
  \author{Nerissa Nance}
  \preauthor{\centering\large\emph}
  \postauthor{\par}
  \predate{\centering\large\emph}
  \postdate{\par}
  \date{November 20, 2018}


\begin{document}
\maketitle

{
\setcounter{tocdepth}{2}
\tableofcontents
}
Citations: A lot of the source material for this training came from
D-lab training materials authored by D-Lab staff. The original content
can be accessed here:
{[}\url{https://github.com/dlab-berkeley/R-Fundamentals}{]} D-Lab works
with Berkeley faculty, research staff, and students to advance
data-intensive social science and humanities research. They're really
cool, check them out!
\href{http://dlab.berkeley.edu/calendar-node-field-date}{D-Lab
Calendar}\\
\href{http://dlab.berkeley.edu/working-groups}{D-lab Working Groups}\\
Some of this also came from Chris Kennedy's ``Concise R'' series,
accessed \href{https://github.com/ck37/concise-r}{here}

Part 1 learning objectives:\\
* Students will be able to structure and execute essential functions,
including using Rmd files and utilizing read key documentation *
Students will be able to recognize and utilize different data types and
data structures * Students will be able to identify different types of
missing data in R * Students will feel comfortable accessing data from R

\section{Introduction}\label{introduction}

\subsubsection{The basics: Navigating RStudio and
Projects}\label{the-basics-navigating-rstudio-and-projects}

Currently, you have the ``KP-R-Intro-Workshop'' project open (as you can
see in the upper right-hand corner of your screen). Projects are handy
because a project's folder is automatically set as the working
directory, which allows us to not worry about explicitly setting the
working directory. This can be more convenient when working on the same
code across different computers.

We can see recent projects by selecting File -\textgreater{} Recent
Projects in the upper left corner of RStudio, or by clicking the list of
recent projects in the upper right corner. If we click the name of the
project it will switch to that project, or if we click the button it
will open in a new session.

Let's try creating a new project.

\begin{itemize}
\tightlist
\item
  Create a new project: File -\textgreater{} New Project
\item
  Choose new directory or existing directory as preferred.
\item
  Click ``open in new session'' to have multiple RStudio projects open.
\end{itemize}

Easy Challenge: close your project and then re-open it.

\subsubsection{RStudio interface}\label{rstudio-interface}

When we start RStudio we have no files open by default. Click File
-\textgreater{} New File -\textgreater{} RScript to start a new R script
(or program). This is a text file with all of the R commands you want to
run for an analysis.

As an alternative to an R script we can use a \textbf{markdown file}. R
Markdown files combine R scripts with simple text formatting, which
makes reports (PDFs or HTMLs) easy to generate with embedded R output.

Try making a new R script and R markdown right now in your new project
directory, and save the files.

RStudio items to note:

\begin{itemize}
\tightlist
\item
  Files show up as tabs in the editor window, usually the upper left.
\item
  The console is the R session where our commands are executed.
\item
  Environment tab in the upper right, and history tab.
\item
  Files tab in bottom right, plus plots, help, etc.
\item
  We can click the ``run'' button to run the current R line in the
  editor.
\end{itemize}

\subsubsection{Basic R markdown file.}\label{basic-r-markdown-file.}

File -\textgreater{} New -\textgreater{} R Markdown

To write R code we need to start an R block with three backticks
followed by r in braces (squiggly brackets - \{ and \}). We can write
commands inside of those brackets to change settings.

Task: everyone google ``RMarkdown cheat sheet'' and save that PDF.

Challenge:

\begin{itemize}
\tightlist
\item
  Create a header line called ``R Training Day 1'', with size 1 (aka
  ``header 1'').
\item
  Write this equation in the notes area: Area = pi * r\^{}2
\item
  In the first chunk of code, write an equation (eg. 2 + 2)
\item
  Compile as a PDF (or html if you prefer).
\end{itemize}

\subsubsection{Package installation}\label{package-installation}

Let's install some packages that we will use during the training. You
can install packages by clicking the ``Packages'' tab in the
bottom-right window, clicking install, and then searching for the
package you wish to install.

Packages can also be installed through the command line using
\texttt{install.packages()}. The package name must be wrapped in
quotation marks so that R knows it is searching for that particular
package named \texttt{"psych"}, rather than previously defined data
named psych:

\begin{quote}
It is generally good to keep RStudio and your packages up to date!
Install the packages we will use in the workshop:
\end{quote}

\begin{Shaded}
\begin{Highlighting}[]
\KeywordTok{install.packages}\NormalTok{(}\KeywordTok{c}\NormalTok{(}\StringTok{"ggplot2"}\NormalTok{, }\StringTok{"knitr"}\NormalTok{, }\StringTok{"psych"}\NormalTok{, }\StringTok{"rmarkdown"}\NormalTok{, }\StringTok{"reshape2"}\NormalTok{, }\StringTok{"swirl"}\NormalTok{, }\StringTok{"mlbench"}\NormalTok{, }\StringTok{"dplyr"}\NormalTok{, }\StringTok{"tidyr"}\NormalTok{))}
\end{Highlighting}
\end{Shaded}

\subsubsection{\texorpdfstring{The \texttt{library}
function}{The library function}}\label{the-library-function}

Before using a previously installed package, you must retreive it with
\texttt{library()} when you begin a new R session. You do not need to
reinstall packages each time you quit and restart your R session.

\begin{Shaded}
\begin{Highlighting}[]
\KeywordTok{library}\NormalTok{(ggplot2)}
\KeywordTok{library}\NormalTok{(knitr)}
\KeywordTok{library}\NormalTok{(psych)}
\end{Highlighting}
\end{Shaded}

\begin{verbatim}
## 
## Attaching package: 'psych'
\end{verbatim}

\begin{verbatim}
## The following objects are masked from 'package:ggplot2':
## 
##     %+%, alpha
\end{verbatim}

\begin{Shaded}
\begin{Highlighting}[]
\KeywordTok{library}\NormalTok{(rmarkdown)}
\KeywordTok{library}\NormalTok{(reshape2)}
\KeywordTok{library}\NormalTok{(swirl)}
\end{Highlighting}
\end{Shaded}

\begin{verbatim}
## 
## | Hi! Type swirl() when you are ready to begin.
\end{verbatim}

\begin{Shaded}
\begin{Highlighting}[]
\CommentTok{# these packages can now be used in our RStudio session!}
\end{Highlighting}
\end{Shaded}

\subsubsection{Getting help}\label{getting-help}

The \texttt{?} symbol can be used to bring up the help pages:

\begin{Shaded}
\begin{Highlighting}[]
\NormalTok{?update.packages}
\NormalTok{?read.csv}
\NormalTok{?summary}
\NormalTok{?describe}
\NormalTok{?plot}
\NormalTok{?ggplot}
\NormalTok{?geometric.mean}
\end{Highlighting}
\end{Shaded}

\subsubsection{\texorpdfstring{Variable assignment
(\texttt{\textless{}-})}{Variable assignment (\textless{}-)}}\label{variable-assignment--}

Variables are special data structures that allow you to enter data into
R. Objects are stored in R's memory and can be retrieved (``called'')
when you need them.

You define objects through \texttt{variable\ assignment} and they are
stored in your \texttt{global\ environment}.

You define objects through variable assignment using the
\textbf{assignment operator} \texttt{\textless{}-}. This is a ``less
than'' \texttt{\textless{}} symbol immediately followed by a hyphen
\texttt{-}

\begin{quote}
You will also see that \texttt{=} is frequently used in place of the
assignment operator. This works the same, but we recommend using the
assignment operator for consistency. Arguments and functions frequently
use \texttt{=} and it is best to consistently assign your variables with
\texttt{\textless{}-} until you understand the basics.
\end{quote}

\textbf{Object definition/variable assignment requires three pieces of
information}:\\
1) object\_name\\
2) \texttt{\textless{}-}\\
3) definition/assignment

Let's define an object named \texttt{numeric\_oject} and define it as
the number 4:

\begin{Shaded}
\begin{Highlighting}[]
\NormalTok{numeric_object <-}\StringTok{ }\DecValTok{4}

\CommentTok{# the ls() function will show us what is in our global environment}
\KeywordTok{ls}\NormalTok{()}
\end{Highlighting}
\end{Shaded}

\begin{verbatim}
## [1] "numeric_object"
\end{verbatim}

\begin{Shaded}
\begin{Highlighting}[]
\NormalTok{r <-}\StringTok{ }\DecValTok{2}

\CommentTok{#back to our equation:}

\NormalTok{Area <-}\StringTok{ }\NormalTok{pi }\OperatorTok{*}\StringTok{ }\NormalTok{r}\OperatorTok{^}\DecValTok{2}

\NormalTok{Area}
\end{Highlighting}
\end{Shaded}

\begin{verbatim}
## [1] 12.56637
\end{verbatim}

\subsubsection{The R global environment}\label{the-r-global-environment}

We now have an object defined in our global environment!

``Call'' (retrieve) the data contained wihtin the object by typing its
name into your script and running the line of code

\begin{Shaded}
\begin{Highlighting}[]
\NormalTok{numeric_object }
\end{Highlighting}
\end{Shaded}

\begin{verbatim}
## [1] 4
\end{verbatim}

\begin{Shaded}
\begin{Highlighting}[]
\KeywordTok{print}\NormalTok{(numeric_object)}
\end{Highlighting}
\end{Shaded}

\begin{verbatim}
## [1] 4
\end{verbatim}

\begin{Shaded}
\begin{Highlighting}[]
\CommentTok{# 4 is returned}
\end{Highlighting}
\end{Shaded}

Let's try another example, this time using character data. Note that
character data is \textbf{always} contained within quotation marks
\texttt{"\ "}

\begin{Shaded}
\begin{Highlighting}[]
\NormalTok{welcome_object <-}\StringTok{ "Welcome"}
\KeywordTok{ls}\NormalTok{()}
\end{Highlighting}
\end{Shaded}

\begin{verbatim}
## [1] "Area"           "numeric_object" "r"              "welcome_object"
\end{verbatim}

We now have four objects in our global environment. Call the data
contained within the object by typing its name and running the line of
code:

\begin{Shaded}
\begin{Highlighting}[]
\NormalTok{welcome_object}
\end{Highlighting}
\end{Shaded}

\begin{verbatim}
## [1] "Welcome"
\end{verbatim}

\subsubsection{Naming R variables}\label{naming-r-variables}

Object names can be anything, but are always case sensitive. However,
they cannot begin with a number and generally do not begin with symbols.
However you choose to name your objects, be consistent and use brief
descriptions of their contents.

Names must be \textbf{\textbf{unique}}. If you reuse an old name, the
old definition will be overwritten.

Let's overwrite the object \texttt{welcome\_object} from above.

\begin{Shaded}
\begin{Highlighting}[]
\NormalTok{welcome_object <-}\StringTok{ "Welcome to class"} 
\NormalTok{welcome_object}
\end{Highlighting}
\end{Shaded}

\begin{verbatim}
## [1] "Welcome to class"
\end{verbatim}

\begin{Shaded}
\begin{Highlighting}[]
\NormalTok{welcome_object <-}\StringTok{ "Welcome to Part 1"}
\KeywordTok{ls}\NormalTok{() }
\end{Highlighting}
\end{Shaded}

\begin{verbatim}
## [1] "Area"           "numeric_object" "r"              "welcome_object"
\end{verbatim}

See how the definition of \texttt{welcome\_object} changed in your
global environment window? However, there are still only two objects in
your global environment.

\subsubsection{\texorpdfstring{Variable classes
(\texttt{class})}{Variable classes (class)}}\label{variable-classes-class}

Each variable in R has a \texttt{class} that summarizes the type of the
data stored within the object. We will talk more about data types below.

\begin{quote}
NOTE! ``numeric'' is the default data type for numbers in R:
\end{quote}

\begin{Shaded}
\begin{Highlighting}[]
\CommentTok{# this is numeric (NOT integer) data type!}
\KeywordTok{class}\NormalTok{(numeric_object) }
\end{Highlighting}
\end{Shaded}

\begin{verbatim}
## [1] "numeric"
\end{verbatim}

\begin{Shaded}
\begin{Highlighting}[]
\CommentTok{# character type}
\KeywordTok{class}\NormalTok{(welcome_object) }
\end{Highlighting}
\end{Shaded}

\begin{verbatim}
## [1] "character"
\end{verbatim}

\subsubsection{\texorpdfstring{Removing variables
(\texttt{rm})}{Removing variables (rm)}}\label{removing-variables-rm}

We remove individual variables from our environment with \texttt{rm()}.
For example, to remove \texttt{numeric\_object}, we type:

\begin{Shaded}
\begin{Highlighting}[]
\KeywordTok{rm}\NormalTok{(numeric_object)}
\KeywordTok{ls}\NormalTok{()}
\end{Highlighting}
\end{Shaded}

\begin{verbatim}
## [1] "Area"           "r"              "welcome_object"
\end{verbatim}

See how \texttt{numeric\_object} disappeared from our global
environment?

We can also wipe the entire environment with \texttt{rm(list\ =\ ls())}
(or, click the broom icon in the upper right ``global environment''
pane)

\begin{Shaded}
\begin{Highlighting}[]
\KeywordTok{rm}\NormalTok{(}\DataTypeTok{list =} \KeywordTok{ls}\NormalTok{()) }
\KeywordTok{ls}\NormalTok{()}
\end{Highlighting}
\end{Shaded}

\begin{verbatim}
## character(0)
\end{verbatim}

Now, all objects are gone from our global environment.

\subsubsection{Vignettes}\label{vignettes}

Double question marks ?? will lead you to coding walkthroughs called
``vignettes''. These usually come with preloaded data and examples:

\begin{Shaded}
\begin{Highlighting}[]
\NormalTok{??psych}
\end{Highlighting}
\end{Shaded}

\begin{verbatim}
## starting httpd help server ... done
\end{verbatim}

You will often find that your questions are not answered by the help
pages nor vignettes. In that case you should Google-search your question
or error message along with the name of a free help website.

For example, to get help making boxplots, I might Google search ``R how
to make boxplots stack overflow''

\section{Atomic data types in R}\label{atomic-data-types-in-r}

Numeric, character, and logical (aka ``boolean'') data types all exist
at the \texttt{atomic\ level}. Normally this means that they cannot be
broken down any further and are the raw inputs for functions in R. Other
R variables are frequently built upon these atomic types.

\subsubsection{Numeric type}\label{numeric-type}

Numeric data are numbers and integers. You may also hear numeric data
referred to as \texttt{float} or \texttt{double} data types. By default,
R stores everything as \texttt{doubles} (64 bit floating point numbers)
which makes R very memory hungry.

Define an object called \texttt{num} and then check its class

\begin{Shaded}
\begin{Highlighting}[]
\NormalTok{num <-}\StringTok{ }\DecValTok{2} \OperatorTok{*}\StringTok{ }\NormalTok{pi}
\NormalTok{num}
\end{Highlighting}
\end{Shaded}

\begin{verbatim}
## [1] 6.283185
\end{verbatim}

\begin{Shaded}
\begin{Highlighting}[]
\KeywordTok{class}\NormalTok{(num)}
\end{Highlighting}
\end{Shaded}

\begin{verbatim}
## [1] "numeric"
\end{verbatim}

Standard mathematical operators apply to the creation of numeric data:
\texttt{+} \texttt{-} \texttt{*} \texttt{\^{}} \texttt{**} \texttt{/}
\texttt{\%*\%\ (matrix\ multiplication)}
\texttt{\%/\%\ (integer\ division)} \texttt{\%\%\ (modular\ division)}

\begin{Shaded}
\begin{Highlighting}[]
\DecValTok{5} \OperatorTok{+}\StringTok{ }\DecValTok{2}  
\end{Highlighting}
\end{Shaded}

\begin{verbatim}
## [1] 7
\end{verbatim}

\begin{Shaded}
\begin{Highlighting}[]
\DecValTok{5} \OperatorTok{-}\StringTok{ }\DecValTok{2}  
\end{Highlighting}
\end{Shaded}

\begin{verbatim}
## [1] 3
\end{verbatim}

\begin{Shaded}
\begin{Highlighting}[]
\DecValTok{5} \OperatorTok{*}\StringTok{ }\DecValTok{2}  
\end{Highlighting}
\end{Shaded}

\begin{verbatim}
## [1] 10
\end{verbatim}

\begin{Shaded}
\begin{Highlighting}[]
\DecValTok{5} \OperatorTok{^}\StringTok{ }\DecValTok{2}  
\end{Highlighting}
\end{Shaded}

\begin{verbatim}
## [1] 25
\end{verbatim}

\begin{Shaded}
\begin{Highlighting}[]
\CommentTok{# same as ^ :}
\DecValTok{5} \OperatorTok{**}\StringTok{ }\DecValTok{2}  
\end{Highlighting}
\end{Shaded}

\begin{verbatim}
## [1] 25
\end{verbatim}

\begin{Shaded}
\begin{Highlighting}[]
\DecValTok{5} \OperatorTok{/}\StringTok{ }\DecValTok{2}  
\end{Highlighting}
\end{Shaded}

\begin{verbatim}
## [1] 2.5
\end{verbatim}

\begin{Shaded}
\begin{Highlighting}[]
\DecValTok{5} \OperatorTok\StringTok{ }\DecValTok{2} 
\end{Highlighting}
\end{Shaded}

\begin{verbatim}
## [1] 2
\end{verbatim}

\begin{Shaded}
\begin{Highlighting}[]
\DecValTok{5} \OperatorTok\StringTok{ }\DecValTok{2}  
\end{Highlighting}
\end{Shaded}

\begin{verbatim}
## [1] 1
\end{verbatim}

\begin{Shaded}
\begin{Highlighting}[]
\DecValTok{5} \OperatorTok\StringTok{ }\DecValTok{2}  
\end{Highlighting}
\end{Shaded}

\begin{verbatim}
##      [,1]
## [1,]   10
\end{verbatim}

\subsubsection{Character type}\label{character-type}

Character (aka string or text) data are always contained within
quotation marks \texttt{"\ "}. Character handling in R is fairly close
to character handling in a Unix terminal.

Let's create an object called \texttt{char} and define it with some
character data. Then, check its class:

\begin{Shaded}
\begin{Highlighting}[]
\NormalTok{char <-}\StringTok{ "This is character data"}
\NormalTok{char}
\end{Highlighting}
\end{Shaded}

\begin{verbatim}
## [1] "This is character data"
\end{verbatim}

\begin{Shaded}
\begin{Highlighting}[]
\KeywordTok{class}\NormalTok{(char)}
\end{Highlighting}
\end{Shaded}

\begin{verbatim}
## [1] "character"
\end{verbatim}

\subsubsection{\texorpdfstring{Contatenating text
(\texttt{paste})}{Contatenating text (paste)}}\label{contatenating-text-paste}

Use \texttt{paste()} to combine/concatenate character data. This will
paste together separate words to larger texts.

\begin{Shaded}
\begin{Highlighting}[]
\NormalTok{char2 <-}\StringTok{ }\KeywordTok{paste}\NormalTok{(}\StringTok{"Hey"}\NormalTok{, }\StringTok{"momma"}\NormalTok{, }\StringTok{"I'm"}\NormalTok{, }\StringTok{"a"}\NormalTok{, }\StringTok{"string"}\NormalTok{)}
\NormalTok{char2}
\end{Highlighting}
\end{Shaded}

\begin{verbatim}
## [1] "Hey momma I'm a string"
\end{verbatim}

Note here that R is not a zero-indexed language - lists begin at the
number 1. The one and only element in this object is the sentence ``Hey
momma I'm a string''.

Character data have some useful functions such as \texttt{substr} and
\texttt{strsplit}. We'll use these to explore function arguments. Use
your help commands to learn more about the commands!

\subsubsection{Function arguments}\label{function-arguments}

Arguments go inside of the parentheses of R functions. Unnamed arguments
are things like the number 4. \texttt{split\ =\ "\ "} is what is called
a named argument. This argument goes inside the parentheses of another
function such as \texttt{strsplit}.

\begin{Shaded}
\begin{Highlighting}[]
\NormalTok{char4 <-}\StringTok{ "This/is/slash/separated/text"}
\NormalTok{char4}
\end{Highlighting}
\end{Shaded}

\begin{verbatim}
## [1] "This/is/slash/separated/text"
\end{verbatim}

\begin{Shaded}
\begin{Highlighting}[]
\NormalTok{split1 <-}\StringTok{ }\KeywordTok{strsplit}\NormalTok{(char4, }\DataTypeTok{split =} \StringTok{"/"}\NormalTok{)}
\NormalTok{split1 }\CommentTok{# a list is returned. }
\end{Highlighting}
\end{Shaded}

\begin{verbatim}
## [[1]]
## [1] "This"      "is"        "slash"     "separated" "text"
\end{verbatim}

Most functions require one or two arguments to be defined in order for
it to properly run. You will find that functions are full of default,
positional, and optional arguments that you can manipulate.

\subsubsection{\texorpdfstring{Simple string subset example
(\texttt{substr})}{Simple string subset example (substr)}}\label{simple-string-subset-example-substr}

\texttt{substr} lets you extract text from certain character positions
in character data. Refer back to \texttt{char2}

\begin{Shaded}
\begin{Highlighting}[]
\NormalTok{char2 <-}\StringTok{ }\KeywordTok{paste}\NormalTok{(}\StringTok{"Hey"}\NormalTok{, }\StringTok{"momma"}\NormalTok{, }\StringTok{"I'm"}\NormalTok{, }\StringTok{"a"}\NormalTok{, }\StringTok{"string"}\NormalTok{)}
\NormalTok{char2}
\end{Highlighting}
\end{Shaded}

\begin{verbatim}
## [1] "Hey momma I'm a string"
\end{verbatim}

If we want to extract just the first four characters of the
\texttt{char2} object we type:

\begin{Shaded}
\begin{Highlighting}[]
\KeywordTok{substr}\NormalTok{(char2, }\DataTypeTok{start =} \DecValTok{1}\NormalTok{, }\DataTypeTok{stop =} \DecValTok{4}\NormalTok{)}
\end{Highlighting}
\end{Shaded}

\begin{verbatim}
## [1] "Hey "
\end{verbatim}

``Hey'' (Hey + blankspace) is returned.

You can use \texttt{substr()} and the assignment operator
\texttt{\textless{}-} to redefine the first four characters in
\texttt{char2} with the word ``Yes'' followed by a blankspace

\begin{Shaded}
\begin{Highlighting}[]
\KeywordTok{substr}\NormalTok{(char2, }\DecValTok{1}\NormalTok{, }\DecValTok{4}\NormalTok{) <-}\StringTok{ "Yes "}
\NormalTok{char2 }
\end{Highlighting}
\end{Shaded}

\begin{verbatim}
## [1] "Yes momma I'm a string"
\end{verbatim}

What changed?

\subsubsection{Logical type}\label{logical-type}

Logical (boolean) data are dichotomous TRUE/FALSE values. R internally
stores \texttt{FALSE} as 0 and \texttt{TRUE} as 1. Define a logical
object:

\begin{Shaded}
\begin{Highlighting}[]
\NormalTok{bool_object <-}\StringTok{ }\OtherTok{TRUE}
\NormalTok{bool_object}
\end{Highlighting}
\end{Shaded}

\begin{verbatim}
## [1] TRUE
\end{verbatim}

\begin{Shaded}
\begin{Highlighting}[]
\KeywordTok{class}\NormalTok{(bool_object)}
\end{Highlighting}
\end{Shaded}

\begin{verbatim}
## [1] "logical"
\end{verbatim}

We recommend to always spell out \texttt{TRUE} and \texttt{FALSE}
instead of abbreviating them \texttt{T} and \texttt{F} which might be
mistaken for previously defined variables - therefore, you should not
save anything as \texttt{T}, \texttt{F}, or \texttt{TRUE} or
\texttt{FALSE}! Thus, there are certain words that do not make good
variable names. R has many reserved words that you should avoid using.
See \texttt{?reserved} for more information.

Note that logical data also take on numeric properties. Remember that
\texttt{TRUE} is stored as the numeral 1 and \texttt{FALSE} is stored as
0.

\begin{Shaded}
\begin{Highlighting}[]
\NormalTok{bool_object }\OperatorTok{+}\StringTok{ }\DecValTok{1}
\end{Highlighting}
\end{Shaded}

\begin{verbatim}
## [1] 2
\end{verbatim}

\begin{Shaded}
\begin{Highlighting}[]
\OtherTok{TRUE} \OperatorTok{-}\StringTok{ }\OtherTok{TRUE}
\end{Highlighting}
\end{Shaded}

\begin{verbatim}
## [1] 0
\end{verbatim}

\begin{Shaded}
\begin{Highlighting}[]
\OtherTok{TRUE} \OperatorTok{+}\StringTok{ }\OtherTok{FALSE}
\end{Highlighting}
\end{Shaded}

\begin{verbatim}
## [1] 1
\end{verbatim}

\begin{Shaded}
\begin{Highlighting}[]
\OtherTok{FALSE} \OperatorTok{-}\StringTok{ }\OtherTok{TRUE}
\end{Highlighting}
\end{Shaded}

\begin{verbatim}
## [1] -1
\end{verbatim}

\subsubsection{Logical tests}\label{logical-tests}

Logical tests are helpful in R if you want to check for equivalence.
Logical tests compare two objects and return a logical output. This is
useful for removing missing data and subsetting (you will learn more
about this in Part 2). Note the use of the double equals \texttt{==}
sign.

\begin{Shaded}
\begin{Highlighting}[]
\NormalTok{?}\StringTok{"=="}
\NormalTok{?}\StringTok{"&"}
\NormalTok{?}\StringTok{"|"}
\NormalTok{?}\StringTok{"!"}
\end{Highlighting}
\end{Shaded}

\begin{Shaded}
\begin{Highlighting}[]
\OtherTok{TRUE} \OperatorTok{==}\StringTok{ }\OtherTok{TRUE} \CommentTok{# is equivalent}
\end{Highlighting}
\end{Shaded}

\begin{verbatim}
## [1] TRUE
\end{verbatim}

\begin{Shaded}
\begin{Highlighting}[]
\OtherTok{FALSE} \OperatorTok{==}\StringTok{ }\OtherTok{FALSE}
\end{Highlighting}
\end{Shaded}

\begin{verbatim}
## [1] TRUE
\end{verbatim}

\begin{Shaded}
\begin{Highlighting}[]
\OtherTok{TRUE} \OperatorTok{==}\StringTok{ }\OtherTok{FALSE}
\end{Highlighting}
\end{Shaded}

\begin{verbatim}
## [1] FALSE
\end{verbatim}

\begin{Shaded}
\begin{Highlighting}[]
\OtherTok{TRUE} \OperatorTok{&}\StringTok{ }\OtherTok{TRUE}   \CommentTok{# and }
\end{Highlighting}
\end{Shaded}

\begin{verbatim}
## [1] TRUE
\end{verbatim}

\begin{Shaded}
\begin{Highlighting}[]
\OtherTok{TRUE} \OperatorTok{|}\StringTok{ }\OtherTok{FALSE}  \CommentTok{# or}
\end{Highlighting}
\end{Shaded}

\begin{verbatim}
## [1] TRUE
\end{verbatim}

\begin{Shaded}
\begin{Highlighting}[]
\OtherTok{TRUE} \OperatorTok{!=}\StringTok{ }\OtherTok{FALSE}  \CommentTok{# not}
\end{Highlighting}
\end{Shaded}

\begin{verbatim}
## [1] TRUE
\end{verbatim}

\begin{Shaded}
\begin{Highlighting}[]
\OtherTok{TRUE} \OperatorTok{>}\StringTok{ }\OtherTok{FALSE} \CommentTok{# greater than}
\end{Highlighting}
\end{Shaded}

\begin{verbatim}
## [1] TRUE
\end{verbatim}

\begin{Shaded}
\begin{Highlighting}[]
\OtherTok{FALSE} \OperatorTok{>=}\StringTok{ }\OtherTok{TRUE} \CommentTok{# greater than or equal to}
\end{Highlighting}
\end{Shaded}

\begin{verbatim}
## [1] FALSE
\end{verbatim}

\section{\texorpdfstring{\textbf{Challenge
1}}{Challenge 1}}\label{challenge-1}

\begin{enumerate}
\def\labelenumi{\arabic{enumi}.}
\tightlist
\item
  What is the three-piece recipe for variable definition? What piece
  goes where?\\
\item
  Define two numeric variables.\\
\item
  Define two character variables.\\
\item
  How do you check what types of data these variables are?\\
\item
  What does the \texttt{ls()} function do? Where else do you see this
  information?\\
\item
  Remove one of your numeric and one of your character variables.\\
\item
  Try to subract your remaining character variable from your numeric
  one. What happens? What might this tell you about data of different
  types?\\
\item
  Define a numeric object as 0 and check its class.\\
\item
  Define a boolean object as ``FALSE'' and check its class.\\
\item
  Use \texttt{==} to compare your numeric and boolean object. Are they
  equivalent? Why or why not?\\
\item
  Define a character object that uses \texttt{paste()} to combine more
  than one word into a sentence.\\
\item
  Use \texttt{substr} to extract the first word of your sentence.\\
\item
  Wipe your environment clean.
\end{enumerate}

\begin{Shaded}
\begin{Highlighting}[]
\NormalTok{## YOUR CODE HERE}
\end{Highlighting}
\end{Shaded}

\begin{quote}
See the below tables to help you remember your logical operations and
functions.
\end{quote}

\href{https://us.sagepub.com/en-us/nam/an-introduction-to-r-for-spatial-analysis-and-mapping/book241031}{Brunsdon
and Comber 2015, page 15} offer a useful summary table of logical
operators in R. These are useful for comparing two data variables to
each other:

\begin{longtable}[]{@{}ll@{}}
\toprule
Logical operator & Description\tabularnewline
\midrule
\endhead
== & Equal\tabularnewline
!= & Not equal\tabularnewline
\textgreater{} & Greater than\tabularnewline
\textless{} & Less than\tabularnewline
\textgreater{}= & Greater than or equal\tabularnewline
\textless{}= & Less than or equal\tabularnewline
! & Not (goes in front of other expressions)\tabularnewline
\& & And (combines expressions)\tabularnewline
\texttt{\textbar{}} & Or (combines expressions)\tabularnewline
\bottomrule
\end{longtable}

\href{https://us.sagepub.com/en-us/nam/an-introduction-to-r-for-spatial-analysis-and-mapping/book241031}{Brunsdon
and Comber 2015, page 102} also offer an excellent summary table of
logical functions in R.

\begin{longtable}[]{@{}ll@{}}
\toprule
Logical function & Description\tabularnewline
\midrule
\endhead
any(x) & \texttt{TRUE} if any in a vector of conditions \texttt{x} is
true\tabularnewline
all(x) & \texttt{TRUE} if all of a vector of conditions \texttt{x} is
true\tabularnewline
is.numeric(x) & \texttt{TRUE} if \texttt{x} contains a numeric
value\tabularnewline
is.character(x) & \texttt{TRUE} if \texttt{x} contains a character
value\tabularnewline
is.logical(x) & \texttt{TRUE} if \texttt{x} contains a true or false
value\tabularnewline
\bottomrule
\end{longtable}

Often it is handy to see what types of data you are working with.
Similar to \texttt{class()} we can see what data type an object is with
\texttt{is.type}. A logical response is returned:

\begin{Shaded}
\begin{Highlighting}[]
\KeywordTok{is.numeric}\NormalTok{(num) }
\end{Highlighting}
\end{Shaded}

\begin{verbatim}
## [1] TRUE
\end{verbatim}

\begin{Shaded}
\begin{Highlighting}[]
\KeywordTok{is.logical}\NormalTok{(bool_object)}
\end{Highlighting}
\end{Shaded}

\begin{verbatim}
## [1] TRUE
\end{verbatim}

\begin{Shaded}
\begin{Highlighting}[]
\KeywordTok{is.logical}\NormalTok{(char2)}
\end{Highlighting}
\end{Shaded}

\begin{verbatim}
## [1] FALSE
\end{verbatim}

\begin{Shaded}
\begin{Highlighting}[]
\KeywordTok{class}\NormalTok{(char2)}
\end{Highlighting}
\end{Shaded}

\begin{verbatim}
## [1] "character"
\end{verbatim}

\subsubsection{\texorpdfstring{Data testing and type coercion
(\texttt{as.type})}{Data testing and type coercion (as.type)}}\label{data-testing-and-type-coercion-as.type}

``Type coercion'' refers to changing the data from one type to another.
You can change data types with \texttt{as.} and then the data type you
wish to convert to.\\
\href{https://us.sagepub.com/en-us/nam/an-introduction-to-r-for-spatial-analysis-and-mapping/book241031}{Brunsdon
and Comber 2015, page 102} also offer an excellent summary table of data
types in R.

\begin{longtable}[]{@{}lll@{}}
\toprule
Type & Test & Conversion\tabularnewline
\midrule
\endhead
character & is.character & as.character\tabularnewline
complex & is.complex & as.complex\tabularnewline
double & is.double & as.double\tabularnewline
expression & is.expression & as.expression\tabularnewline
integer & is.integer & as.integer\tabularnewline
list & is.list & as.list\tabularnewline
logical & is.logical & as.logical\tabularnewline
numeric & is.numeric & as.numeric\tabularnewline
single & is.single & as.single\tabularnewline
raw & is.raw & as.raw\tabularnewline
\bottomrule
\end{longtable}

\subsubsection{\texorpdfstring{Type coercion example
(\texttt{as.numeric})}{Type coercion example (as.numeric)}}\label{type-coercion-example-as.numeric}

\begin{Shaded}
\begin{Highlighting}[]
\CommentTok{# Create some character data}
\NormalTok{char_data <-}\StringTok{ "9"}
\KeywordTok{class}\NormalTok{(char_data)}
\end{Highlighting}
\end{Shaded}

\begin{verbatim}
## [1] "character"
\end{verbatim}

\begin{Shaded}
\begin{Highlighting}[]
\CommentTok{# Coerce this character data to numeric data type}
\KeywordTok{as.numeric}\NormalTok{(char_data)}
\end{Highlighting}
\end{Shaded}

\begin{verbatim}
## [1] 9
\end{verbatim}

\begin{Shaded}
\begin{Highlighting}[]
\KeywordTok{class}\NormalTok{(char_data)}
\end{Highlighting}
\end{Shaded}

\begin{verbatim}
## [1] "character"
\end{verbatim}

\begin{Shaded}
\begin{Highlighting}[]
\CommentTok{# What happened here? Why did `char_data` not change classes? (hint: we did not overwrite the object!)}

\NormalTok{char.data <-}\StringTok{ }\KeywordTok{as.numeric}\NormalTok{(char_data)}
\KeywordTok{class}\NormalTok{(char.data)}
\end{Highlighting}
\end{Shaded}

\begin{verbatim}
## [1] "numeric"
\end{verbatim}

\begin{Shaded}
\begin{Highlighting}[]
\NormalTok{char.data}
\end{Highlighting}
\end{Shaded}

\begin{verbatim}
## [1] 9
\end{verbatim}

\subsubsection{\texorpdfstring{Type coercion example 2
(\texttt{as.integer})}{Type coercion example 2 (as.integer)}}\label{type-coercion-example-2-as.integer}

You can change numeric type to integer type using \texttt{as.integer}

\begin{Shaded}
\begin{Highlighting}[]
\NormalTok{num <-}\StringTok{ }\DecValTok{4}
\KeywordTok{class}\NormalTok{(num)}
\end{Highlighting}
\end{Shaded}

\begin{verbatim}
## [1] "numeric"
\end{verbatim}

\begin{Shaded}
\begin{Highlighting}[]
\NormalTok{int <-}\StringTok{ }\KeywordTok{as.integer}\NormalTok{(num)}
\KeywordTok{class}\NormalTok{(int)}
\end{Highlighting}
\end{Shaded}

\begin{verbatim}
## [1] "integer"
\end{verbatim}

\begin{Shaded}
\begin{Highlighting}[]
\NormalTok{int}
\end{Highlighting}
\end{Shaded}

\begin{verbatim}
## [1] 4
\end{verbatim}

Now, create some character data and try to convert it to integer type:

\begin{Shaded}
\begin{Highlighting}[]
\CommentTok{# Create a new object}
\NormalTok{char.num <-}\StringTok{ "three"}
\NormalTok{char.num   }\CommentTok{#Note that the word three is contained within " " }
\end{Highlighting}
\end{Shaded}

\begin{verbatim}
## [1] "three"
\end{verbatim}

\begin{Shaded}
\begin{Highlighting}[]
\KeywordTok{class}\NormalTok{(char.num)}
\end{Highlighting}
\end{Shaded}

\begin{verbatim}
## [1] "character"
\end{verbatim}

\begin{Shaded}
\begin{Highlighting}[]
\CommentTok{# What happens if we try to coerce character to numeric data type? }
\KeywordTok{as.integer}\NormalTok{(char.num)}
\end{Highlighting}
\end{Shaded}

\begin{verbatim}
## Warning: NAs introduced by coercion
\end{verbatim}

\begin{verbatim}
## [1] NA
\end{verbatim}

Why did this fail? Can R change character data to numbers? Why not?
(hint: R has no protocol for automatically coerce words to numbers). As
you can see, trying to coerce data types can lead to weird behavior
sometimes.

\section{\texorpdfstring{\textbf{Challenge
2}}{Challenge 2}}\label{challenge-2}

\begin{enumerate}
\def\labelenumi{\arabic{enumi}.}
\tightlist
\item
  Create a character object and check its type using
  \texttt{is.character} and \texttt{class}. What is the difference
  between these two methods?\\
\item
  Try to change (``coerce'') this object to another data type using
  \texttt{as.integer}, \texttt{as.numeric}, and \texttt{as.logical}.\\
\item
  Create an object of class ``integer''. Remember, there are two ways to
  do this!\\
\item
  Why is 1 == ``1'' true? Why is -1 \textless{} FALSE true? Why is
  ``one'' \textless{} 2 false?
\end{enumerate}

\begin{Shaded}
\begin{Highlighting}[]
\NormalTok{## YOUR CODE HERE}
\end{Highlighting}
\end{Shaded}

\section{Data structures}\label{data-structures}

There are several kinds of data structures in R. Data structures are
collections of data objects (e.g., numeric, character, and logical
vectors, lists, and matrices) that work together. These four are the
most common:

\begin{enumerate}
\def\labelenumi{\arabic{enumi}.}
\tightlist
\item
  vector\\
\item
  list\\
\item
  matrix\\
\item
  dataframe
\end{enumerate}

\subsubsection{3.1 Vector}\label{vector}

A \textbf{VECTOR} is an ordered group of the same kind of data.
``Ordered'' means that their position matters. Vectors are
one-dimensional and homogenous, and are thus referred to by their type
(e.g., character vector, numeric vector, logical vector).

Create a numeric vector by combining/concatenating elements with
\texttt{c()}

\begin{Shaded}
\begin{Highlighting}[]
\NormalTok{?c}
\end{Highlighting}
\end{Shaded}

\begin{Shaded}
\begin{Highlighting}[]
\NormalTok{numeric_vector <-}\StringTok{ }\KeywordTok{c}\NormalTok{(}\DecValTok{3}\NormalTok{, }\DecValTok{5}\NormalTok{, }\DecValTok{6}\NormalTok{, }\DecValTok{5}\NormalTok{, }\DecValTok{3}\NormalTok{)}
\NormalTok{numeric_vector}
\end{Highlighting}
\end{Shaded}

\begin{verbatim}
## [1] 3 5 6 5 3
\end{verbatim}

This numeric vector contains five elements.

You can index a vector using square brackets (more on this in the
subsetting section of Part 2). For example, to see what value lives in
the third position of the object \texttt{numeric\_vector}, you could
type:

\begin{Shaded}
\begin{Highlighting}[]
\NormalTok{numeric_vector[}\DecValTok{3}\NormalTok{]}
\end{Highlighting}
\end{Shaded}

\begin{verbatim}
## [1] 6
\end{verbatim}

You can also add items to a vector using \texttt{c()} and a comma
\texttt{,} (as long as it is the same data type)

\begin{Shaded}
\begin{Highlighting}[]
\NormalTok{numeric_vector2 <-}\StringTok{ }\KeywordTok{c}\NormalTok{(numeric_vector, }\DecValTok{78}\NormalTok{)}
\NormalTok{numeric_vector2}
\end{Highlighting}
\end{Shaded}

\begin{verbatim}
## [1]  3  5  6  5  3 78
\end{verbatim}

It doesn't matter what the datatype is for a vector, as long as it is
all the same

\begin{Shaded}
\begin{Highlighting}[]
\NormalTok{logical_vector <-}\StringTok{ }\KeywordTok{c}\NormalTok{(}\OtherTok{TRUE}\NormalTok{, }\OtherTok{TRUE}\NormalTok{, }\OtherTok{FALSE}\NormalTok{, }\OtherTok{FALSE}\NormalTok{, }\OtherTok{TRUE}\NormalTok{)}
\NormalTok{logical_vector}
\end{Highlighting}
\end{Shaded}

\begin{verbatim}
## [1]  TRUE  TRUE FALSE FALSE  TRUE
\end{verbatim}

\begin{Shaded}
\begin{Highlighting}[]
\NormalTok{logical_vector2 <-}\StringTok{ }\KeywordTok{c}\NormalTok{(logical_vector, }\KeywordTok{c}\NormalTok{(}\OtherTok{FALSE}\NormalTok{, }\OtherTok{FALSE}\NormalTok{, }\OtherTok{FALSE}\NormalTok{))}
\NormalTok{logical_vector2}
\end{Highlighting}
\end{Shaded}

\begin{verbatim}
## [1]  TRUE  TRUE FALSE FALSE  TRUE FALSE FALSE FALSE
\end{verbatim}

You can also add and multiply vectors, but they need to be the same
length

\begin{Shaded}
\begin{Highlighting}[]
\NormalTok{logical_vector }\OperatorTok{*}\StringTok{ }\NormalTok{logical_vector}
\end{Highlighting}
\end{Shaded}

\begin{verbatim}
## [1] 1 1 0 0 1
\end{verbatim}

\subsubsection{\texorpdfstring{Generating random vectors
(\texttt{seq})}{Generating random vectors (seq)}}\label{generating-random-vectors-seq}

You might need to create vectors that are sequences of numbers. You can
do this via \texttt{seq}. Here we specify a vector from zero to the
\texttt{length} of our object \texttt{logical\_vector2} (eight). The
argument \texttt{by\ =\ 2} tells R that we want only the even numbers!

\begin{Shaded}
\begin{Highlighting}[]
\NormalTok{?length}
\NormalTok{logical_vector2}
\KeywordTok{length}\NormalTok{(logical_vector2)}

\KeywordTok{seq}\NormalTok{(}\DataTypeTok{from=}\DecValTok{0}\NormalTok{,}\DataTypeTok{to=}\KeywordTok{length}\NormalTok{(logical_vector2),}\DataTypeTok{by=}\DecValTok{2}\NormalTok{)}
\end{Highlighting}
\end{Shaded}

\subsubsection{\texorpdfstring{Sequence where \texttt{by\ =\ 1}
(\texttt{:})}{Sequence where by = 1 (:)}}\label{sequence-where-by-1}

R also gives you a shorthand operator for creating sequences in whole
number increments of 1. This is the colon symbol \texttt{:}

\begin{Shaded}
\begin{Highlighting}[]
\DecValTok{0}\OperatorTok{:}\DecValTok{8}
\end{Highlighting}
\end{Shaded}

\begin{verbatim}
## [1] 0 1 2 3 4 5 6 7 8
\end{verbatim}

\begin{Shaded}
\begin{Highlighting}[]
\NormalTok{sequence_object <-}\StringTok{ }\DecValTok{28}\OperatorTok{:}\DecValTok{36}
\NormalTok{sequence_object }
\end{Highlighting}
\end{Shaded}

\begin{verbatim}
## [1] 28 29 30 31 32 33 34 35 36
\end{verbatim}

\begin{Shaded}
\begin{Highlighting}[]
\DecValTok{0}\OperatorTok{:}\KeywordTok{length}\NormalTok{(logical_vector2)}
\end{Highlighting}
\end{Shaded}

\begin{verbatim}
## [1] 0 1 2 3 4 5 6 7 8
\end{verbatim}

\subsubsection{\texorpdfstring{\texttt{set.seed}}{set.seed}}\label{set.seed}

You can also sample random groups of numbers. Use \texttt{set.seed()} to
ensure that we all always get the same random draws from the parent
universe, even on different machines

\begin{Shaded}
\begin{Highlighting}[]
\NormalTok{?set.seed}
\NormalTok{?runif}
\NormalTok{?rnorm}
\NormalTok{?sample}
\end{Highlighting}
\end{Shaded}

Set the seed, and then choose our values:

\begin{Shaded}
\begin{Highlighting}[]
\KeywordTok{set.seed}\NormalTok{(}\DecValTok{2346}\NormalTok{)}
\CommentTok{# 20 random samples from uniform distribution between the numbers 3 and 7}
\NormalTok{uniform <-}\StringTok{ }\KeywordTok{runif}\NormalTok{(}\DecValTok{20}\NormalTok{, }\DecValTok{3}\NormalTok{, }\DecValTok{7}\NormalTok{) }
\NormalTok{uniform}
\end{Highlighting}
\end{Shaded}

\begin{verbatim}
##  [1] 4.544240 3.489348 6.068407 3.161751 6.461309 5.810518 3.404867
##  [8] 5.238931 6.954126 6.328505 3.048943 6.305050 4.382296 4.475999
## [15] 3.133005 3.390534 4.084209 4.689436 5.267931 4.996230
\end{verbatim}

\begin{Shaded}
\begin{Highlighting}[]
\CommentTok{# 20 random samples from the normal distribution with a mean of 0 and standard deviation of 1}
\NormalTok{normal <-}\StringTok{ }\KeywordTok{rnorm}\NormalTok{(}\DecValTok{20}\NormalTok{, }\DecValTok{0}\NormalTok{, }\DecValTok{1}\NormalTok{) }
\NormalTok{normal}
\end{Highlighting}
\end{Shaded}

\begin{verbatim}
##  [1] -0.09149019  1.12753115 -1.61933145 -0.48718366 -1.06576577
##  [6] -0.74445302 -2.35303871  0.15443995  0.35509938 -0.06067869
## [11]  0.31884664  1.32218276  1.19400064 -0.23740576 -2.15278944
## [16]  0.80304167 -0.31741855 -1.35466867  0.03503021 -0.81104751
\end{verbatim}

\begin{Shaded}
\begin{Highlighting}[]
\CommentTok{# 20 random samples from between the numbers 5 and 10. Note that `replace=TRUE` signifies that it is OK to reuse numbers already selected.}
\NormalTok{integer <-}\StringTok{ }\KeywordTok{sample}\NormalTok{(}\DecValTok{5}\OperatorTok{:}\DecValTok{10}\NormalTok{, }\DecValTok{20}\NormalTok{, }\DataTypeTok{replace =} \OtherTok{TRUE}\NormalTok{) }
\NormalTok{integer}
\end{Highlighting}
\end{Shaded}

\begin{verbatim}
##  [1]  7 10  7  5  7 10  8  6  6  7  8  5  8  5  5  7  9  9 10  7
\end{verbatim}

\begin{Shaded}
\begin{Highlighting}[]
\NormalTok{character <-}\StringTok{ }\KeywordTok{sample}\NormalTok{(}\KeywordTok{c}\NormalTok{(}\StringTok{"Cat"}\NormalTok{, }\StringTok{"Dog"}\NormalTok{, }\StringTok{"Pig"}\NormalTok{), }\DecValTok{20}\NormalTok{, }\DataTypeTok{replace =} \OtherTok{TRUE}\NormalTok{) }\CommentTok{# 20 random samples of character data}
\NormalTok{character}
\end{Highlighting}
\end{Shaded}

\begin{verbatim}
##  [1] "Cat" "Cat" "Pig" "Pig" "Dog" "Pig" "Cat" "Dog" "Pig" "Dog" "Cat"
## [12] "Pig" "Dog" "Pig" "Pig" "Cat" "Pig" "Cat" "Pig" "Cat"
\end{verbatim}

\subsubsection{List}\label{list}

A \textbf{LIST} is an ordered group of data that are not of the same
type. Lists are heterogenous. Instead of using \texttt{c()} like in
vector creation, use \texttt{list()} to create a list:

\begin{Shaded}
\begin{Highlighting}[]
\NormalTok{?list}

\NormalTok{list1 <-}\StringTok{ }\KeywordTok{list}\NormalTok{(}\OtherTok{TRUE}\NormalTok{, }\StringTok{"one"}\NormalTok{, }\DecValTok{1}\NormalTok{) }\CommentTok{# include three kinds of data: logical, character, and integer}
\NormalTok{list1}
\KeywordTok{class}\NormalTok{(list1)}
\end{Highlighting}
\end{Shaded}

Lists are simple containers and are not additive or multiplicative like
vectors and matrices are:

\begin{Shaded}
\begin{Highlighting}[]
\NormalTok{list1 }\OperatorTok{*}\StringTok{ }\KeywordTok{list}\NormalTok{(}\OtherTok{FALSE}\NormalTok{, }\StringTok{"zero"}\NormalTok{, }\DecValTok{0}\NormalTok{) }\CommentTok{# Error}
\end{Highlighting}
\end{Shaded}

\subsubsection{Matrix}\label{matrix}

\textbf{MATRICES} are homogenous like vectors. They are tables comprised
of data all of the same type. Matrices are organized into rows and
columns.

Use \texttt{matrix()} to create a matrix

\begin{Shaded}
\begin{Highlighting}[]
\NormalTok{?matrix}
\end{Highlighting}
\end{Shaded}

We can also specify how we want the matrix to be organized using the
\texttt{nrow} and \texttt{ncol} arguments:

\begin{Shaded}
\begin{Highlighting}[]
\NormalTok{matrix1 <-}\StringTok{ }\KeywordTok{matrix}\NormalTok{(}\DecValTok{1}\OperatorTok{:}\DecValTok{12}\NormalTok{, }\DataTypeTok{nrow =} \DecValTok{4}\NormalTok{, }\DataTypeTok{ncol =} \DecValTok{3}\NormalTok{) }\CommentTok{# this makes a 4 x 3 matrix}
\NormalTok{matrix1}
\end{Highlighting}
\end{Shaded}

\begin{verbatim}
##      [,1] [,2] [,3]
## [1,]    1    5    9
## [2,]    2    6   10
## [3,]    3    7   11
## [4,]    4    8   12
\end{verbatim}

\begin{Shaded}
\begin{Highlighting}[]
\KeywordTok{class}\NormalTok{(matrix1)}
\end{Highlighting}
\end{Shaded}

\begin{verbatim}
## [1] "matrix"
\end{verbatim}

We can also coerce a vector to a matrix, because a vector is comprised
of homogenous data of the same kind, just like a matrix is:

\begin{Shaded}
\begin{Highlighting}[]
\CommentTok{# Create a numeric vector from 1 to 20}
\NormalTok{vec1 <-}\StringTok{ }\KeywordTok{c}\NormalTok{(}\DecValTok{1}\OperatorTok{:}\DecValTok{20}\NormalTok{)}
\NormalTok{vec1}
\end{Highlighting}
\end{Shaded}

\begin{verbatim}
##  [1]  1  2  3  4  5  6  7  8  9 10 11 12 13 14 15 16 17 18 19 20
\end{verbatim}

\begin{Shaded}
\begin{Highlighting}[]
\KeywordTok{class}\NormalTok{(vec1)}
\end{Highlighting}
\end{Shaded}

\begin{verbatim}
## [1] "integer"
\end{verbatim}

\begin{Shaded}
\begin{Highlighting}[]
\CommentTok{# Coerce this vector to a matrix with 10 rows and 2 columns:}
\NormalTok{matrix2 <-}\StringTok{ }\KeywordTok{matrix}\NormalTok{(vec1, }\DataTypeTok{ncol=}\DecValTok{2}\NormalTok{)}
\NormalTok{matrix2}
\end{Highlighting}
\end{Shaded}

\begin{verbatim}
##       [,1] [,2]
##  [1,]    1   11
##  [2,]    2   12
##  [3,]    3   13
##  [4,]    4   14
##  [5,]    5   15
##  [6,]    6   16
##  [7,]    7   17
##  [8,]    8   18
##  [9,]    9   19
## [10,]   10   20
\end{verbatim}

\begin{Shaded}
\begin{Highlighting}[]
\KeywordTok{class}\NormalTok{(matrix2)}
\end{Highlighting}
\end{Shaded}

\begin{verbatim}
## [1] "matrix"
\end{verbatim}

\subsubsection{Data frame}\label{data-frame}

\emph{\textbf{It is worth emphasizing the importance of data frames in
R!}} Inside R, a \textbf{DATA FRAME} is a list of equal-length vectors.
These vectors can be of different types. Data frames are thus
heterogenous like lists.

This is simply a spreadsheet!

Let's create a dataframe called \texttt{animals} using the vectors we
already created:

We do this using \texttt{data.frame()}

\begin{Shaded}
\begin{Highlighting}[]
\NormalTok{?data.frame}
\end{Highlighting}
\end{Shaded}

\begin{Shaded}
\begin{Highlighting}[]
\NormalTok{animals <-}\StringTok{ }\KeywordTok{data.frame}\NormalTok{(uniform, normal, integer, character, }\DataTypeTok{stringsAsFactors =} \OtherTok{FALSE}\NormalTok{)}
\CommentTok{# }\AlertTok{NOTE}\CommentTok{: `stringsAsFactors=FALSE` means that R will NOT try to interpret character data as factor type. More on this below. }
\end{Highlighting}
\end{Shaded}

Take a peek at the \texttt{animals} data frame to see what it looks
like:

\begin{Shaded}
\begin{Highlighting}[]
\KeywordTok{head}\NormalTok{(animals)}
\end{Highlighting}
\end{Shaded}

\begin{verbatim}
##    uniform      normal integer character
## 1 4.544240 -0.09149019       7       Cat
## 2 3.489348  1.12753115      10       Cat
## 3 6.068407 -1.61933145       7       Pig
## 4 3.161751 -0.48718366       5       Pig
## 5 6.461309 -1.06576577       7       Dog
## 6 5.810518 -0.74445302      10       Pig
\end{verbatim}

We can change the names of the columns by passing into it a vector with
our desired names

\begin{Shaded}
\begin{Highlighting}[]
\CommentTok{# Create a vector called `new_df_names` with the new column names and pass this vector into `colnames()`}
\KeywordTok{colnames}\NormalTok{(animals) <-}\StringTok{ }\KeywordTok{c}\NormalTok{(}\StringTok{"Weight"}\NormalTok{, }\StringTok{"Progress"}\NormalTok{, }\StringTok{"Height"}\NormalTok{, }\StringTok{"Type"}\NormalTok{)}
\KeywordTok{head}\NormalTok{(animals)}
\end{Highlighting}
\end{Shaded}

\begin{verbatim}
##     Weight    Progress Height Type
## 1 4.544240 -0.09149019      7  Cat
## 2 3.489348  1.12753115     10  Cat
## 3 6.068407 -1.61933145      7  Pig
## 4 3.161751 -0.48718366      5  Pig
## 5 6.461309 -1.06576577      7  Dog
## 6 5.810518 -0.74445302     10  Pig
\end{verbatim}

We can check the structure of our data frame via \texttt{str()}

\begin{Shaded}
\begin{Highlighting}[]
\NormalTok{?str}
\KeywordTok{str}\NormalTok{(animals)}
\end{Highlighting}
\end{Shaded}

Here, we can see that a data frame is just a list of equal-length
vectors! For readability purposes, \texttt{str()} displays columns from
top to bottom, while the data are displayed left to right.

\subparagraph{Learning about your data
frame}\label{learning-about-your-data-frame}

\begin{Shaded}
\begin{Highlighting}[]
\CommentTok{# View the dimensions (nrow x ncol) of the data frame:}
\KeywordTok{dim}\NormalTok{(animals) }
\end{Highlighting}
\end{Shaded}

\begin{verbatim}
## [1] 20  4
\end{verbatim}

\begin{Shaded}
\begin{Highlighting}[]
\CommentTok{# View column names:}
\KeywordTok{colnames}\NormalTok{(animals)}
\end{Highlighting}
\end{Shaded}

\begin{verbatim}
## [1] "Weight"   "Progress" "Height"   "Type"
\end{verbatim}

\begin{Shaded}
\begin{Highlighting}[]
\CommentTok{# View row names (we did not change these, so they default to character type)}
\KeywordTok{rownames}\NormalTok{(animals)}
\end{Highlighting}
\end{Shaded}

\begin{verbatim}
##  [1] "1"  "2"  "3"  "4"  "5"  "6"  "7"  "8"  "9"  "10" "11" "12" "13" "14"
## [15] "15" "16" "17" "18" "19" "20"
\end{verbatim}

\begin{Shaded}
\begin{Highlighting}[]
\KeywordTok{class}\NormalTok{(}\KeywordTok{rownames}\NormalTok{(animals))}
\end{Highlighting}
\end{Shaded}

\begin{verbatim}
## [1] "character"
\end{verbatim}

\subparagraph{Factor data type}\label{factor-data-type}

Factor data are categorical types used to make comparisons between other
data. Factors group the data by their ``levels'' (the different
categorical groups within a factor).

For example, in our \texttt{animals} dataframe, we can coerce ``Type''
from character to factor data type. Cat, Dog, and Pig are the factor
levels. If we might want to compare heights and weights between Cat,
Dog, and Pigs, we set the character ``Type'' vector to factor data type.
We can do so with \texttt{factor()}.

The dollar sign operator \texttt{\$} is used to call a single vector
from a data frame. This will be discussed more in Part 2 along with the
rest of subsetting.

\begin{Shaded}
\begin{Highlighting}[]
\CommentTok{# "Name" is character data type. See how each column name is preceded by `$`?}
\KeywordTok{str}\NormalTok{(animals)   }
\end{Highlighting}
\end{Shaded}

\begin{verbatim}
## 'data.frame':    20 obs. of  4 variables:
##  $ Weight  : num  4.54 3.49 6.07 3.16 6.46 ...
##  $ Progress: num  -0.0915 1.1275 -1.6193 -0.4872 -1.0658 ...
##  $ Height  : int  7 10 7 5 7 10 8 6 6 7 ...
##  $ Type    : chr  "Cat" "Cat" "Pig" "Pig" ...
\end{verbatim}

\begin{Shaded}
\begin{Highlighting}[]
 \KeywordTok{class}\NormalTok{(animals}\OperatorTok{$}\NormalTok{Type)}
\end{Highlighting}
\end{Shaded}

\begin{verbatim}
## [1] "character"
\end{verbatim}

\begin{Shaded}
\begin{Highlighting}[]
\NormalTok{animals}\OperatorTok{$}\NormalTok{Type <-}\StringTok{ }\KeywordTok{factor}\NormalTok{(animals}\OperatorTok{$}\NormalTok{Type)}
\CommentTok{# "Name" is now factor data type!}
\KeywordTok{str}\NormalTok{(animals)  }
\end{Highlighting}
\end{Shaded}

\begin{verbatim}
## 'data.frame':    20 obs. of  4 variables:
##  $ Weight  : num  4.54 3.49 6.07 3.16 6.46 ...
##  $ Progress: num  -0.0915 1.1275 -1.6193 -0.4872 -1.0658 ...
##  $ Height  : int  7 10 7 5 7 10 8 6 6 7 ...
##  $ Type    : Factor w/ 3 levels "Cat","Dog","Pig": 1 1 3 3 2 3 1 2 3 2 ...
\end{verbatim}

Notice that R stores factors internally as integers and uses the
character strings as labels. Also notice that by default R orders
factors alphabetically and returns them when we call the ``Type''
vector.

\begin{Shaded}
\begin{Highlighting}[]
\NormalTok{animals}\OperatorTok{$}\NormalTok{Type}
\end{Highlighting}
\end{Shaded}

\begin{verbatim}
##  [1] Cat Cat Pig Pig Dog Pig Cat Dog Pig Dog Cat Pig Dog Pig Pig Cat Pig
## [18] Cat Pig Cat
## Levels: Cat Dog Pig
\end{verbatim}

\begin{Shaded}
\begin{Highlighting}[]
\KeywordTok{levels}\NormalTok{(animals}\OperatorTok{$}\NormalTok{Type)}
\end{Highlighting}
\end{Shaded}

\begin{verbatim}
## [1] "Cat" "Dog" "Pig"
\end{verbatim}

\subparagraph{\texorpdfstring{Changing factor
``levels''}{Changing factor levels}}\label{changing-factor-levels}

Each animal type (Cat, Dog, and Pig) within the factor
\texttt{animals\$Type} vector are the factor levels.

If we want to change how R stores the factor levels, we can specify
their levels using the \texttt{levels()} argument. For example:

\begin{Shaded}
\begin{Highlighting}[]
\NormalTok{animals}\OperatorTok{$}\NormalTok{Type  }
\end{Highlighting}
\end{Shaded}

\begin{verbatim}
##  [1] Cat Cat Pig Pig Dog Pig Cat Dog Pig Dog Cat Pig Dog Pig Pig Cat Pig
## [18] Cat Pig Cat
## Levels: Cat Dog Pig
\end{verbatim}

\begin{Shaded}
\begin{Highlighting}[]
\CommentTok{# What if we want to change the factor level sort to Levels: Dog Pig Cat?}
\NormalTok{animals}\OperatorTok{$}\NormalTok{Type <-}\StringTok{ }\KeywordTok{factor}\NormalTok{(animals}\OperatorTok{$}\NormalTok{Type, }\DataTypeTok{levels =} \KeywordTok{c}\NormalTok{(}\StringTok{"Dog"}\NormalTok{, }\StringTok{"Pig"}\NormalTok{, }\StringTok{"Cat"}\NormalTok{))}

\CommentTok{# Now when we call animals$Name, we can see that the levels have changed}
\NormalTok{animals}\OperatorTok{$}\NormalTok{Type }
\end{Highlighting}
\end{Shaded}

\begin{verbatim}
##  [1] Cat Cat Pig Pig Dog Pig Cat Dog Pig Dog Cat Pig Dog Pig Pig Cat Pig
## [18] Cat Pig Cat
## Levels: Dog Pig Cat
\end{verbatim}

\begin{quote}
NOTE: we will load the \texttt{animals} data frame from file at the
beginning of Part 2, so do not worry if your dataframe does not look
identical!
\end{quote}

\section{Importing and Exporting}\label{importing-and-exporting}

\subsection{\texorpdfstring{Importing csv data files using
\texttt{read.csv}}{Importing csv data files using read.csv}}\label{importing-csv-data-files-using-read.csv}

Check first where your working directory is by typing:

\begin{Shaded}
\begin{Highlighting}[]
\KeywordTok{getwd}\NormalTok{()}
\end{Highlighting}
\end{Shaded}

\begin{verbatim}
## [1] "I:/R training/KP-R-Intro-Workshop"
\end{verbatim}

(note: since we're using R projects, you should already know where your
wd is, because it's set up automatically).

You can import data using the read.csv command. Take a moment to look at
the help file for syntax tips:

\begin{Shaded}
\begin{Highlighting}[]
\NormalTok{?read.csv}
\end{Highlighting}
\end{Shaded}

\begin{Shaded}
\begin{Highlighting}[]
\NormalTok{df.who <-}\StringTok{ }\KeywordTok{read.csv}\NormalTok{(}\StringTok{"./data/who_suicide_statistics.csv"}\NormalTok{, }
                    \DataTypeTok{header =} \OtherTok{TRUE}\NormalTok{, }
                    \DataTypeTok{stringsAsFactors =} \OtherTok{FALSE}\NormalTok{,}
                    \DataTypeTok{na.strings =} \StringTok{" "}\NormalTok{)}
\KeywordTok{dim}\NormalTok{(df.who)}
\end{Highlighting}
\end{Shaded}

\begin{verbatim}
## [1] 43776     6
\end{verbatim}

\begin{Shaded}
\begin{Highlighting}[]
\KeywordTok{head}\NormalTok{(df.who)}
\end{Highlighting}
\end{Shaded}

\begin{verbatim}
##   country year    sex         age suicides_no population
## 1 Albania 1985 female 15-24 years          NA     277900
## 2 Albania 1985 female 25-34 years          NA     246800
## 3 Albania 1985 female 35-54 years          NA     267500
## 4 Albania 1985 female  5-14 years          NA     298300
## 5 Albania 1985 female 55-74 years          NA     138700
## 6 Albania 1985 female   75+ years          NA      34200
\end{verbatim}

\begin{Shaded}
\begin{Highlighting}[]
\KeywordTok{str}\NormalTok{(df.who)}
\end{Highlighting}
\end{Shaded}

\begin{verbatim}
## 'data.frame':    43776 obs. of  6 variables:
##  $ country    : chr  "Albania" "Albania" "Albania" "Albania" ...
##  $ year       : int  1985 1985 1985 1985 1985 1985 1985 1985 1985 1985 ...
##  $ sex        : chr  "female" "female" "female" "female" ...
##  $ age        : chr  "15-24 years" "25-34 years" "35-54 years" "5-14 years" ...
##  $ suicides_no: int  NA NA NA NA NA NA NA NA NA NA ...
##  $ population : int  277900 246800 267500 298300 138700 34200 301400 264200 296700 325800 ...
\end{verbatim}

\begin{quote}
Notice that \texttt{stringsAsFactors\ =\ FALSE}. If set to
\texttt{TRUE}, R will try to guess which \textbf{character} vectors
should automatically be converted to factors. This is problematic
because 1) R is not always good at guessing and 2) R defaults to
alphabetical and increasing numeric factor level sorting. This might not
matter for your data, but it is recommended to set
\texttt{stringsAsFactors\ =\ FALSE} and manually convert your desired
character vectors to factors.
\end{quote}

\begin{quote}
\texttt{header\ =\ TRUE} will include the header row;
\texttt{header\ =\ FALSE} will turn your header row into the first row
of actual data. \texttt{na.strings\ =\ c("\ ",\ 999)} indicates that
blank cells and cells coded 999 allows you to specify data that should
be automatically converted to \texttt{NA} upon importation. We do not
have any here, so nothing is altered.
\end{quote}

\subsection{Importing and exporting with the Rio
package}\label{importing-and-exporting-with-the-rio-package}

The Rio package simplifies importing/exporting by putting all
data-specific import tools under one umbrella (popularly known as the
``swiss army knife'' of data importing and exporting) See
\href{https://cran.r-project.org/web/packages/rio/vignettes/rio.html}{this
helpful rio documentation as well}.

\begin{Shaded}
\begin{Highlighting}[]
\KeywordTok{install.packages}\NormalTok{(}\StringTok{"rio"}\NormalTok{)}
\NormalTok{?rio}
\end{Highlighting}
\end{Shaded}

\begin{Shaded}
\begin{Highlighting}[]
\KeywordTok{library}\NormalTok{(rio)}

\NormalTok{df.who2 <-}\StringTok{ }\KeywordTok{import}\NormalTok{(}\StringTok{"./data/whostats_suicide.sas7bdat"}\NormalTok{)}
 \KeywordTok{dim}\NormalTok{(df.who2)}
\end{Highlighting}
\end{Shaded}

\begin{verbatim}
## [1] 43776     6
\end{verbatim}

\begin{Shaded}
\begin{Highlighting}[]
 \KeywordTok{head}\NormalTok{(df.who)}
\end{Highlighting}
\end{Shaded}

\begin{verbatim}
##   country year    sex         age suicides_no population
## 1 Albania 1985 female 15-24 years          NA     277900
## 2 Albania 1985 female 25-34 years          NA     246800
## 3 Albania 1985 female 35-54 years          NA     267500
## 4 Albania 1985 female  5-14 years          NA     298300
## 5 Albania 1985 female 55-74 years          NA     138700
## 6 Albania 1985 female   75+ years          NA      34200
\end{verbatim}

\begin{Shaded}
\begin{Highlighting}[]
\CommentTok{#Are these dataframes the same? Let's see:}
 \KeywordTok{all.equal}\NormalTok{(df.who,df.who2)}
\end{Highlighting}
\end{Shaded}

\begin{verbatim}
##  [1] "Attributes: < Names: 1 string mismatch >"                                  
##  [2] "Attributes: < Length mismatch: comparison on first 2 components >"         
##  [3] "Attributes: < Component 2: Modes: numeric, character >"                    
##  [4] "Attributes: < Component 2: Lengths: 43776, 1 >"                            
##  [5] "Attributes: < Component 2: target is numeric, current is character >"      
##  [6] "Component \"country\": Attributes: < target is NULL, current is list >"    
##  [7] "Component \"year\": Attributes: < target is NULL, current is list >"       
##  [8] "Component \"sex\": Attributes: < target is NULL, current is list >"        
##  [9] "Component \"age\": Attributes: < target is NULL, current is list >"        
## [10] "Component \"suicides_no\": Modes: numeric, character"                      
## [11] "Component \"suicides_no\": Attributes: < target is NULL, current is list >"
## [12] "Component \"suicides_no\": target is numeric, current is character"        
## [13] "Component \"population\": Attributes: < target is NULL, current is list >"
\end{verbatim}

\begin{Shaded}
\begin{Highlighting}[]
 \KeywordTok{class}\NormalTok{(df.who}\OperatorTok{$}\NormalTok{suicides_no)}
\end{Highlighting}
\end{Shaded}

\begin{verbatim}
## [1] "integer"
\end{verbatim}

\begin{Shaded}
\begin{Highlighting}[]
\KeywordTok{class}\NormalTok{(df.who2}\OperatorTok{$}\NormalTok{suicides_no)}
\end{Highlighting}
\end{Shaded}

\begin{verbatim}
## [1] "character"
\end{verbatim}

\section{\texorpdfstring{\textbf{Challenge
3}}{Challenge 3}}\label{challenge-3}

\begin{enumerate}
\def\labelenumi{\arabic{enumi}.}
\tightlist
\item
  Save the ``animals'' data frame as a .csv file named ``animals.csv''.
  How many basic arguments should you use to save using the
  \texttt{write.csv} function?
\item
  Import the ``animals'' data frame using the rio package.
\item
  Compare the two data frames. Are they the same? Why or why not?
\end{enumerate}

\begin{Shaded}
\begin{Highlighting}[]
\NormalTok{## YOUR CODE HERE}
\end{Highlighting}
\end{Shaded}

\section{Visualize data (teaser)}\label{visualize-data-teaser}

We'll use this more in parts 2 and 3, but take a moment to look at the
data you imported. You can use the \texttt{hist()} command to do a
histogram and the \texttt{summary()} command to look at summary
statistics. For example:

\begin{Shaded}
\begin{Highlighting}[]
\KeywordTok{names}\NormalTok{(df.who)}
\end{Highlighting}
\end{Shaded}

\begin{verbatim}
## [1] "country"     "year"        "sex"         "age"         "suicides_no"
## [6] "population"
\end{verbatim}

\begin{Shaded}
\begin{Highlighting}[]
\KeywordTok{hist}\NormalTok{(df.who}\OperatorTok{$}\NormalTok{suicides_no)}
\end{Highlighting}
\end{Shaded}

\includegraphics{KP_R_Intro_Part1_files/figure-latex/plotting-1.pdf}

\begin{Shaded}
\begin{Highlighting}[]
\KeywordTok{summary}\NormalTok{(df.who}\OperatorTok{$}\NormalTok{suicides_no)}
\end{Highlighting}
\end{Shaded}

\begin{verbatim}
##    Min. 1st Qu.  Median    Mean 3rd Qu.    Max.    NA's 
##     0.0     1.0    14.0   193.3    91.0 22338.0    2256
\end{verbatim}

\begin{Shaded}
\begin{Highlighting}[]
\KeywordTok{plot}\NormalTok{(df.who}\OperatorTok{$}\NormalTok{year, df.who}\OperatorTok{$}\NormalTok{suicides)}
\end{Highlighting}
\end{Shaded}

\includegraphics{KP_R_Intro_Part1_files/figure-latex/plotting-2.pdf}

\section{\texorpdfstring{\textbf{Challenge 4} Homework
(optional)}{Challenge 4 Homework (optional)}}\label{challenge-4-homework-optional}

swirl is a package that helps you learn R by using R. Visit the
\href{http://swirlstats.com/}{swirl} homepage to learn more

\begin{Shaded}
\begin{Highlighting}[]
\KeywordTok{library}\NormalTok{(swirl)}
\KeywordTok{swirl}\NormalTok{()}
\CommentTok{# follow the instructions until you can select number 1 "R Programming: The basics of programming in R"}
\end{Highlighting}
\end{Shaded}

\begin{quote}
NOTE: type \texttt{bye()} to exist swirl.
\end{quote}

Acknowledgements - \href{http://adv-r.had.co.nz/}{Wickham H. 2014.
Advanced R}\\
- \href{http://www.jaredlander.com/r-for-everyone/}{Lander J. 2013. R
for everyone: Advanced analytics and graphics}\\
- \href{https://www.nostarch.com/artofr.htm}{Matloff N. 2011. The art of
R programming: A tour of statistical software design}\\
-
\href{https://us.sagepub.com/en-us/nam/an-introduction-to-r-for-spatial-analysis-and-mapping/book241031}{Brunsdon
C, Comber L. 2015. An Introduction to R for Spatial Analysis and
Mapping} - Contributions by Dlab affiliates Evan Muzzall, Rochelle
Terman, Dillon Niederhut, Sam Abdel-Ghaffar


\end{document}
